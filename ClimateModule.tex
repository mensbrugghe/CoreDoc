\subsection{The carbon cycle}

The carbon cycle described herein is derived
from \cite{MillaretalACP2017} and \cite{SmithetalGMD2018}
as implemented by \cite{Haensel2020} and 
is known as the \emph{Finite Amplitude Impulse Response} (FAIR) model. 

The carbon cycle depends on total \COT{} emissions:
fossil-fuel combustion-based, other industrial \COT{} emissions, $\mathit{EmiX}^C$,
and changes in land-use emissions, $\mathit{EmiLand}^C$. For
the moment, the latter two sources of \COT{} emissions are
exogenous. Total global \COT{} emissions are provided
in equation~(\ref{eq:ECO2}). 

The carbon cycle also depends
on cumulative carbon emissions---expressed in gtC.
Equation~(\ref{eq:cume}) represents the standard
cumulative expression, where industrial emissions
are converted from gtCO2 to gtC.
Equation~(\ref{eq:cumetot}) takes into account
the exogenous cumulative emissions from land-use changes.
The starting point for 2015 is 400 gtC of cumulative
industrial emissions and 197 gtC of cumulative land-use
emissions.\footnote{\cite{Haensel2020} use a value of 197 gtC, which
is nearly double the level in the 2016 version of the DICE model, see
\cite{Nordhaus2016NBERw22933}.}

\begin{equation}
\label{eq:ECO2}
\mathit{Emi}^C_t = \mathit{EmiGbl}_{\mathit{co2},t}
 + \mathit{EmiX}^C_t + \mathit{EmiLand}^C_t
\end{equation}

\begin{equation}
\label{eq:cume}
\mathit{CumEmiInd}_t = \mathit{CumEmiInd}_{t-1} 
+ (12/44) \left(\mathit{EmiGbl}_{\mathit{co2},t-1}
 + \mathit{EmiX}^C_{t-1} \right)
\end{equation}

\begin{equation}
\label{eq:cumetot}
\mathit{CumEmi}^T_t = \mathit{CumEmiInd}_t + \mathit{CumEmiLand}_t
\end{equation}

The FAIR model uses anthropogenic fossil and land use \COT{}
emissions as input and partitions them into four boxes $R_b$---with
partition fraction $\phi$ and $\tau$ representing the
differing timescales of carbon uptake by geological processes
(\texttt{GEOPR}), the deep ocean (\texttt{DPOCN}), the biosphere (\texttt{BIOSF}) and the ocean
mixed layer (\texttt{MXOCN}). The one-step motion equation
is given by equation~(\ref{eq:r1}). The carbon reservoir receives a
fraction of the \COT{} emissions impulse and it is adjusted by some
level of 'decay'.\footnote{Appendix~\ref{chap:AppDyn} describes
the multi-step version of the carbon cycle.}
The $\alpha$ parameter is endogenous and is calibrated dynamically so
that the overall behavior of the carbon cycle mimics other earth systems models (ESMs). This is detailed further below.

\begin{equation}
\label{eq:r1}
\Delta \mathit{R}_{b,t} = \phi_b (12/44) E_{t-1} - \displaystyle \frac {\mathit{R}_{b,t-1}}{\alpha \tau_b} \iff \mathit{R}_{b,t} = \mathit{R}_{b,t-1} \left(\displaystyle 1 -\frac {1}{\alpha \tau_b} \right) + \phi_b (12/44) \mathit{Emi}^C_{t-1}
\end{equation}

The calibration of $\alpha$ is based
on \cite{MillaretalACP2017}:
"To identify a suitable state dependence, we focus on parameterising
variations in the 100-year integrated impulse response
function, $\mathit{iIRF100}$. A focus on the integrated impulse
response (average airborne fraction over a period of time,
multiplied by the length of time period), as opposed to the
airborne fraction at a particular point in time, is more closely
related to the impact of \COT{} emissions on the global energy
budget, and also to other metrics such as global warming potential.
With other coefficients fixed, iIRF100 is a monotonic
(but nonlinear) function of $\alpha$." 
Equation~(\ref{eq:iIRF100A}) expresses this relation.

In addition from the same
authors, "We assume $\mathit{iIRF100}$ is a function of the accumulated
perturbation carbon stock in the land and ocean (equivalent
to the amount of emitted carbon that no longer resides in
the atmosphere), ..., and of the global
mean temperature anomaly from preindustrial conditions, $T^{\mathit{atm}}$."
Equation~(\ref{eq:iIRF100B}) expresses this relation,
where $\mathit{MAT}$ is the carbon stock residing in the atmosphere.
In the model code, the two expressions for $\mathit{iIRF100}$
are substituted out and this relation determines the $\alpha$ parameter
dynamically. Finally, equation~({\ref{eq:mat}}) defines the atmospheric concentration (in gTC)
for each time period.

\begin{equation}
\label{eq:iIRF100A}
\mathit{iIRF100} = \displaystyle \sum_b{\alpha_t \phi_b \tau_b \left[
1 - \exp\left( -\frac{100}{\alpha_t \tau_b}\right)\right] }
\end{equation}

\begin{equation}
\label{eq:iIRF100B}
\mathit{iIRF100} = r_0 + r_C\left(\mathit{CumEmi}^T_t - (\mathit{MAT}_t - 588)\right)
+ r_T T^{\mathit{atm}}_t
\end{equation}

\begin{equation}
\label{eq:mat}
\mathit{MAT}_{t} = \sum_b{\mathit{R}_{b,t}} + 588
\end{equation}

\subsection{Radiative forcing}

Radiative forcing is the sum of the impacts of individual gases on radiative forcing.
In the case of carbon, it takes a logarithmic form where the key parameter is
$\kappa$, which represents the equilibrium level of forcing for a doubling of carbon
concentration relative to its pre-industrial level. The default
parameterization sets $\kappa=3.6813$. At the moment, all other forcings
are collapsed into a single exogenous time-varying level.\footnote{Future model
versions will add endogenous forcings for some of the other gases that
are generated by \textsc{Envisage}.}

\begin{equation}
\mathit{FORC}_{\mathit{CO2},t} = \kappa \frac{\ln\left[\mathit{MAT}_t/\mathit{MAT}^e \right]}{\ln(2)}
\end{equation}


\begin{equation}
\mathit{RF}_{t} = \sum_{\mathit{frc}}{\mathit{FORC}_{\mathit{em},t}} +  \mathit{FORC}^X_{t}
\end{equation}

\subsection{The energy balance model (EBM)}

The energy balance model (EBM) is derived from \cite{LeachetalFAIR2_0GeoModDev2020}.
This EBM has a 3-box model: land and atmosphere (\texttt{ATMOS}), shallow (\texttt{UPOCN}) and deep oceans (\texttt{DPOCN}).
A common representation
of this physical process is outlined by \cite{GeoffroyetalJClim2013A}. 
The \cite{LeachetalFAIR2_0GeoModDev2020} paper extends the 2-box model
to three boxes to better represent the transient response.

The motion equation for the change in temperature across the three
boxes is represented by the following matrix differential equation:

\[
\mathbf{\dot T} = A \mathbf{T} + \mathbf{M} \mathit{RF}
\]

\noindent where $\mathbf{\dot T}'=[\begin{array}{c c c } \dot T_{\mathit{atmos}} & \dot T_{\mathit{upocn}} & \dot T_{\mathit{upocn}}\end{array}]$,
$\mathbf{T'} = [\begin{array}{c c c } T_{\mathit{atmos}} & T_{\mathit{upocn}} & T_{\mathit{upocn}}\end{array}]$, $\mathit{RF}$ is radiative
forcing, $\mathbf{M'} = [\begin{array}{c c c } 1/C_1 & 0 & 0\end{array}]$ and:

\[
A = \left( \begin{array}{c c c }
-\left(\lambda+\kappa_2\right)/C_1 & \kappa_2/C_1 & 0 \\
\kappa_2/C_2 & -\left(\kappa_2 +\varepsilon \kappa_3\right)/C_2 &
\varepsilon \kappa_3/C_2 \\
0 & \kappa_3/C_3 & - \kappa_3/C_3 \\
\end{array} \right)
\]

The '$C$' parameters represent the heat capacity of each of the
boxes. The '$\kappa$' parameters represent the heat exchange
coefficients between the boxes. The '$\lambda$' parameter is called
the climate feedback parameter. And the '$\epsilon$' parameter
is an efficacy factor that affects transient warming. In the
form given in equation~(\ref{eq:TEMP}), the change in atmospheric temperature
can be solved as a difference equation, which also includes
the temperature of the two ocean layers.\footnote{In the FAIR model, the EBM model
is converted to an impulse-response model that only solves for the change
in atmospheric temperature. In essence, this involves diagonalizing
the $A$ matrix by calculating the eigenvalues and eigenvectors. Both
the EBM and IR versions yield very similar results. Results available
from the author.}

\begin{equation}
\label{eq:TEMP}
\mathbf{Temp}_t = (I+A)\mathbf{Temp}_{t-1} + \mathbf{M} \mathit{RF}_t
\end{equation}

There are different strategies for converting
these equations for use in models with varying time
steps. In the new GAMS code, the model
implements equation~(\ref{eq:TEMP}) using an annual time step,
i.e. the time dimension for this
equation differs from the time step
of most of the other equations. The only modification
is for the forcing impulsion. In the current
code, the average forcing is used for any individual
year between two model periods. For example, if
the year is 2023, the relevant forcing level
is the average of 2020 and 2025 if using
5-year time steps from 2015 forwards. Forcing is
thus uniform between two model periods. Given
the relatively slow evolution of forcing, the
choice is unlikely to have a major impact
especially with relatively small step sizes such
as 5-years.\footnote{The relevant forcing
for any year $\mathit{tt}$ is governed by a mapping
between $\mathit{tt}$ and the model period denoted
by $t$. In the code, this is governed by the mapping named \texttt{mapt}.
It is initialized in the climate data file, for
example \texttt{ClimDataV10} for Version 10
of the GTAP Data Base. The user can
change the flag \texttt{ifFORCAVG} to the
value 0 to use only the contemporaneous
forcing, rather than the average over two
simulation periods.}

Parameterization is based on the latest parameter
values from \cite{LeachetalFAIR2_0GeoModDev2020}.
\cite{Haensel2020} uses an earlier vesion
of FAIR (v1.5), which is based on a two
box model. In that case, $\kappa_3$ is set to zero
and there is no need for $\epsilon$. There is
a different interpretation of the labels. The
label \texttt{ATMOS} represents land, atmosphere and
the shallow oceans and the label \texttt{UPOCN} represents
the deep ocean.