% The CDE description
The \emph{Constant Difference of Elasticities} (CDE) function is a generalization of the CES
function, but it allows for more flexibility in terms of substitution effects across goods and
for non-homotheticity.\footnote{More detailed descriptions of the CDE can be found in
\cite{Herteletal1991}, \cite{Surry1993} and \cite{Hertel1997}.}  The starting point is an
implicitly additive indirect utility function (see \cite{Hanoch1975}) from which we can derive
demand using Roy's identity (and the implicit function theorem).

\subsection{General form}

A dual approach is used to determine the properties of the CDE function. The indirect utility
function is defined implicitly via the following expression:

\begin{equation}
\label{eq:CDEV}
V(p,u,Y) = \sum_{i=1}^{n}{\alpha_iu^{e_i b_i}\left( \frac{p_i}{y}\right)^{b_i}} \equiv 1
\end{equation}

\noindent where $p$ is the vector of commodity prices, $u$ is (per capita) utility and $y$ is
per capita income. Using Roy's identity and the implicit function theorem\footnote{See
\cite{Varian1992}, p. 109.} we can derive demand, $x$, where $v$ is the indirect utility function
(defined implicitly):

\begin{equation}
{x_i} =  - \frac{{\partial v}}{{\partial {p_i}}} /
\frac{{\partial v}}{{\partial Y}} =
- \left( {\frac{{\partial V}}{{\partial {p_i}}} /
\frac{{\partial V}}{{\partial u}}} \right) /
\left( {\frac{{\partial V}}{{\partial Y}} /
\frac{{\partial V}}{{\partial u}}} \right)
=  - \left( {\frac{{\partial V}}{{\partial {p_i}}}/\frac{{\partial V}}{{\partial Y}}} \right)
\end{equation}

\noindent This then leads to the following demand function:

\begin{equation}
\label{eq:CDEP}
{x_i} = \frac{{{\alpha_i}{b_i}{u^{{e_i}{b_i}}}{{\left( {\frac{{{p_i}}}{y}} \right)}^{{b_i} - 1}}}}
{{\sum\limits_j {{\alpha_j}{b_j}{u^{{e_j}{b_j}}}{{\left( {\frac{{{p_j}}}{y}} \right)}^{{b_j}}}} }}
\end{equation}

Implementation is easier if we define the following variable:

\begin{equation}
\mathit{ZC}_i = \alpha_i b_i u^{e_i b_i}\left( \frac{p_i}{y}\right)^{b_i}
\end{equation}

\noindent Then the budget shares can be expressed as:

\begin{equation}
s_i = \frac{\mathit{ZC}_i}{\sum_j{\mathit{ZC}_j}}
\end{equation}

\noindent and the demand expression is:

\begin{equation}
x_i = \frac{s_i}{p_i} y
\end{equation}

\noindent Implementation also requires evaluating $u$. This can be
done by implementing equation~(\ref{eq:CDEV}) and inserting the
expression for $\mathit{ZC}$:

\begin{equation}
\sum_{i=1}^{n}{\frac {\mathit{ZC}_i} {b_i} } \equiv 1
\end{equation}

\subsection{Elasticities}

In order to calibrate the CDE system, it is necessary to derive the demand and income elasticities
of the CDE. The algebra is tedious, but straightforward.

The own-price elasticity is given by the following:

\begin{equation}
\label{eq:EPS0}
\varepsilon_i = \frac{\partial x_i} {\partial p_i} \frac{p_i}{x_i} =
\frac{\displaystyle s_i \left[\sum_j {s_j e_j b_j} - e_i b_i  \right]} 
{\displaystyle \sum_j {s_j e_j}}
+ b_i \left( 1 - s_i \right) - 1
\end{equation}

\noindent In deriving the elasticity, we make use of the following formula that defines the
elasticity of utility with respect to price (and again makes use of the implicit function theorem):

\begin{equation}
\frac{{\partial u}}{{\partial {p_i}}}\frac{{{p_i}}}{u} =  - \frac{{{p_i}}}{u}\left(
{\frac{{\partial V}}{{\partial {p_i}}}} \right)/\left( {\frac{{\partial V}}{{\partial u}}} \right)
=  - \frac{{{s_i}}}{{\sum\limits_j {{s_j}{e_j}} }}
\end{equation}

\noindent The price elasticity of utility is approximately the value share of the respective demand
component as long as the weighted sum of the expansion parameters, $e$, is close to unity.
The value (or budget) share is defined in the next equation:

\begin{equation}
s_i = \frac{p_i x_i}{y}
\end{equation}

\noindent Letting $\sigma_i = 1-b_i$ (or $b_i = 1-\sigma_i$), we can also write:

\begin{equation}
{\varepsilon_i} = {s_i}\left[ {{\sigma_i} - \frac{{{e_i}(1 - {\sigma_i})}}
{{\sum\limits_j {{s_j}{e_j}} }} - \frac{{\sum\limits_j {{s_j}{e_j}{\sigma_j}} }}
{{\sum\limits_j {{s_j}{e_j}} }}} \right] - {\sigma_i}
\end{equation}

\noindent With $\sigma$ uniform, we also have:

\begin{equation}
{\varepsilon_i} =  - \frac{{{s_i}{e_i}(1 - \sigma )}}{{\sum\limits_j {{s_j}{e_j}} }} - \sigma
\end{equation}

\noindent With both $e$ and $\sigma$ uniform, the formula simplifies to:

\begin{equation}
\label{eq:CESELAS}
{\varepsilon_i} =  - {s_i}(1 - \sigma ) - \sigma  = \sigma ({s_i} - 1) - {s_i}
\end{equation}

\noindent Equation~(\ref{eq:CESELAS}) reflects the own-price elasticity for the standard CES
utility function. Finally, with $e$ uniform but not $\sigma$, we have:

\begin{equation}
{\varepsilon_i} = {s_i}\left[ {2{\sigma_i} - 1 - \sum\limits_j {{s_j}{\sigma_j}} } \right]
- {\sigma_i}
\end{equation}

The derivation of the cross elasticities is almost identical and will not be carried out here.
Combining both the own-and cross price elasticities, the matrix of substitution elasticities
takes the following form where we use the Kronecker product, $\delta$:\footnote{$\delta$ takes
the value of 1 along the diagonal (i.e. when $i=j$) and the value 0 off-diagonal
(i.e. when $i \ne j$).}

\begin{equation}
{\varepsilon_{ij}} = {s_j}\left[ { - {b_j} - \frac{{{e_i}{b_i}}}{{\sum\limits_k {{s_k}{e_k}} }}
+ \frac{{\sum\limits_k {{s_k}{e_k}{b_k}} }}{{\sum\limits_k {{s_k}{e_k}} }}} \right]
+ {\delta_{ij}}({b_i} - 1)
\end{equation}

\noindent Again, we replace $b$ by $1-\sigma$, to get:

\begin{equation}
\label{eq:EPS}
{\varepsilon_{ij}} = {s_j}\left[ {{\sigma_j} - \frac{{{e_i}(1 - {\sigma_i})}}
{{\sum\limits_k {{s_k}{e_k}} }} - \frac{{\sum\limits_k {{s_k}{e_k}{\sigma_k}} }}
{{\sum\limits_k {{s_k}{e_k}} }}} \right] - {\delta_{ij}}{\sigma_i}
\end{equation}

\noindent For uniform $\sigma$, equation~(\ref{eq:EPS}) takes the form:

\begin{equation}
{\varepsilon_{ij}} =  - \frac{{{e_i}{s_j}(1 - \sigma )}}
{{\sum\limits_k {{s_k}{e_k}} }} - {\delta_{ij}}\sigma
\end{equation}

\noindent And with a uniform $\sigma$ and $e$, i.e. the CES assumption, we have:

\begin{equation}
{\varepsilon_{ij}} =  - {s_j}(1 - \sigma ) - {\delta_{ij}}\sigma  =
\sigma ({s_j} - {\delta_{ij}}) - {s_j}
\end{equation}

\noindent Finally, for a uniform $e$ only, the matrix of elasticities is:

\begin{equation}
{\varepsilon_{ij}} = {s_j}\left[ {{\sigma_j} - (1 - {\sigma_i}) -
\sum\limits_k {{s_k}{\sigma_k}} } \right] - {\delta_{ij}}{\sigma_i}
\end{equation}

The income elasticities are derived in a similar fashion:

\begin{equation}
\label{eq:ETA}
\eta_i = \frac{\partial {x_i}} {\partial Y} \frac{Y} {{x_i}} =
   \frac {1} {\sum\limits_k {{s_k} {e_k}}}
   \left[ {e_i}{b_i} - \sum\limits_k {s_k} {e_k}{b_k} \right]
   - ({b_i} - 1) + \sum\limits_k {{b_k}{s_k}}
\end{equation}

\noindent For this, we need the elasticity of utility with respect to income:

\begin{equation}
\label{eq:ETAU}
\frac{{\partial u}}{{\partial Y}}\frac{Y}{u} =  - \frac{Y}{u}\left( {\frac{{\partial V}}
{{\partial Y}}} \right)/\left( {\frac{{\partial V}}{{\partial u}}} \right) =
\frac{1}{{\sum\limits_k {{s_k}{e_k}} }}
\end{equation}

\noindent Note that for a uniform and unitary $e$, the income elasticity of utility is~1.

\noindent Replacing $b$ with $1-\sigma$, equation~(\ref{eq:ETA}) can be re-written to be:

\begin{equation}
\label{eq:ETAS}
{\eta_i} = \frac{1}{{\sum\limits_k {{s_k}{e_k}} }}\left[ {{e_i}(1 - {\sigma_i})
+ \sum\limits_k {{s_k}{e_k}{\sigma_k}} } \right] + {\sigma_i} - \sum\limits_k {{s_k}{\sigma_k}}
\end{equation}

\noindent With a uniform $\sigma$, the income elasticity becomes:

\begin{equation}
{\eta_i} = \frac{1}{{\sum\limits_k {{s_k}{e_k}} }}\left[ {{e_i}(1 - \sigma )
+ \sigma \sum\limits_k {{s_k}{e_k}} } \right] = \frac{{{e_i}(1 - \sigma )}}
{{\sum\limits_k {{s_k}{e_k}} }} + \sigma
\end{equation}

\noindent With $e$ uniform, the income elasticity is unitary, irrespective of the values of
the $\sigma$ parameters.

From the Slutsky equation, we can calculate the compensated demand elasticities:

\begin{equation}
{\xi_{ij}} = {\varepsilon_{ij}} + {s_j}{\eta_i} =  - {\delta_{ij}}{\sigma_i}
+ {s_j}\left[ {{\sigma_j} + {\sigma_i} - \sum\limits_k {{s_k}{\sigma_k}} } \right]
\end{equation}

\noindent The cross-Allen partial elasticities are equal to the compensated demand
elasticities divided by the share:

\begin{equation}
\label{eq:APE}
\sigma_{ij}^a = {\sigma_j} + {\sigma_i} - \sum\limits_k {{s_k}{\sigma_k}}
- {\delta_{ij}}{\sigma_i}/{s_j}
\end{equation}

\noindent It can be readily seen that the difference of the partial elasticities is constant,
hence the name of \emph{constant difference in elasticities}.

\begin{equation}
\sigma_{ij}^a - \sigma_{il}^a = {\sigma_j} - {\sigma_l}
\end{equation}

\noindent With a uniform $\sigma$, we revert back to the standard CES where there is equivalence
between the CES substitution elasticity and the cross-Allen partial elasticity:

\begin{equation}
\sigma_{ij}^a = {\sigma}
\end{equation}

\subsection{Calibration of the CDE}

Calibration assumes that we know the budget shares, the own uncompensated demand elasticities and
the income elasticities. The weighted sum of the income elasticities must equal 1, so the first
step in the calibration procedure is to make sure Engel's law holds. One alternative is to fix
some (or none) of the income elasticities and re-scale the others using least squares. The problem
is to minimize the following objective function:

\begin{displaymath}
{\sum\limits_{i \in \Omega } {\left( {{\eta_i} - \eta_i^0} \right)}^2}
\end{displaymath}

\noindent subject to

\begin{displaymath}
\sum\limits_{i \in \Omega } {{s_i}{\eta_i}}  = 1 - \sum\limits_{i \notin \Omega } {{s_i}{\eta_i}}
\end{displaymath}

\noindent where the set $\Omega$ contains all sectors where the income elasticity is not fixed,
i.e. its complement contains those sectors with fixed income elasticities. The solution is:

\begin{displaymath}
\begin{array}{*{20}{c}}
\eta_i = \eta_i^0 + s_i
\frac{\displaystyle {1 - \sum\limits_{j \notin \Omega } {s_j \eta_j}
- \displaystyle \sum\limits_{j \in \Omega } {s_j \eta_j^0} }}
{{\displaystyle \sum\limits_{j \in \Omega } {s_j^2} }}&{\forall i \in \Omega }
\end{array}
\end{displaymath}

Calibration of the $\sigma$ parameters is straightforward given the own elasticities and the input
value shares. The first step is to calculate the Allen partial elasticities, these are simply the
income elasticity adjusted by the own elasticities divided by the budget shares:

\begin{equation}
\label{eq:APEOWN}
\sigma_{ii}^a = {\eta_i} + \frac{{{\varepsilon_{ii}}}}{{{s_i}}}
\end{equation}

Next, equation~(\ref{eq:APE}) is setup in matrix form:

\begin{equation}
\sigma_{ii}^a = A{\sigma_i}
\end{equation}

\noindent where the matrix $A$ has the form:
\begin{equation}
A = \left[
{\begin{array}{*{20}{c}}
\displaystyle  {2 - \frac {1}{{{s_1}}} - {s_1}}&{ - {s_2}}& \ldots &{ - {s_n}}\\
{ - {s_1}} & \displaystyle {2 - \frac{1}{{{s_2}}} - {s_2}}& \ldots &{ - {s_n}}\\
 \vdots & \vdots & \ddots & \vdots \\
{ - {s_1}}&{ - {s_2}}& \ldots & \displaystyle {2 - \frac{1}{{{s_n}}} - {s_n}}
\end{array}} \right]
\end{equation}

\noindent or each element of $A$ has the following formula:

\begin{displaymath}
{a_{ij}} = {\delta_{ij}}(2 - 1/{s_i}) - {s_j}
\end{displaymath}

\noindent We can then solve for $\sigma$ (and back-out the $b$ parameters):

\begin{equation}
\label{eq:SIGMACAL}
{\sigma_i} = {A^{ - 1}}\sigma_{ii}^a
\end{equation}

\noindent There is nothing which guarantees the consistency of the calibrated $\sigma$ parameters,
which are meant to be positive. The calculation of the $\sigma$ parameters depends only on the
budget shares and the own-price uncompensated elasticities. If the calibrated $\sigma$ parameters
are not all positive, one could modify the elasticities until consistency is achieved. In practice,
problems have occurred when a sector's budget share dominates total expenditure.

The $e$ parameters are derived from Equation~(\ref{eq:ETAS}) and normalizing them so that their
share weighted sum is equal to 1. Equation~(\ref{eq:ETAS}) can then be converted to matrix form
and inverted:

\begin{equation}
B = \left[ {\begin{array}{*{20}{c}}
{{s_1}{\sigma_1} + (1 - {\sigma_1})}&{{s_2}{\sigma_2}}& \ldots &{{s_n}{\sigma_n}}\\
{{s_1}{\sigma_1}}&{{s_2}{\sigma_2} + (1 - {\sigma_2})}& \ldots &{{s_n}{\sigma_n}}\\
 \vdots & \vdots & \ddots & \vdots \\
{{s_1}{\sigma_1}}&{{s_2}{\sigma_2}}& \ldots &{{s_n}{\sigma_n} + (1 - {\sigma_n})}
\end{array}} \right]
\end{equation}

\noindent or

\begin{displaymath}
{b_{ij}} = {s_j}{\sigma_j} + {\delta_{ij}}(1 - {\sigma_i})
\end{displaymath}

\noindent Then the $e$ parameters are derived from matrix inversion:

\begin{equation}
\label{eq:ECAL}
{e_i} = {B^{ - 1}}{C_i} = {B^{ - 1}}\left( {{\eta_i} - {\sigma_i}
+ \sum\limits_k {{s_k}{\sigma_k}} } \right)
\end{equation}

Calibration of the $\alpha$ parameters is based on equations~(\ref{eq:CDEV}) and~(\ref{eq:CDEP}).
Start first with equation~(\ref{eq:CDEP}) and write it in terms relative to consumption
of good~1, i.e.:

\begin{equation}
\frac{{{x_i}}}{{{x_1}}} =
\frac {\displaystyle {{\alpha_i}{b_i}{u^{{e_i}{b_i}}}
{{\left( {\frac{\displaystyle {{p_i}}}{\displaystyle Y}} \right)}^{{b_i} - 1}}}}
{\displaystyle {{\alpha_1}{b_1}{u^{{e_1}{b_1}}}
{{\left( {\frac{\displaystyle {{p_1}}}{\displaystyle Y}} \right)}^{{b_1} - 1}}}}
\end{equation}

\noindent This equation can be used to isolate $\alpha_i$:

\begin{equation}
\label{eq:alphaCal1}
{\alpha_i} = \frac{\displaystyle {{x_i}}}{\displaystyle {{x_1}}}
\frac{\displaystyle {{\alpha_1}{b_1} {\displaystyle u^{{e_1}{b_1}}}
{{\left( {\frac{\displaystyle {{p_1}}}{\displaystyle Y}} \right)}^{{b_1} - 1}}}}
{\displaystyle {{b_i}{u^{{e_i}{b_i}}}
{{\left( {\frac{\displaystyle {{p_i}}}{\displaystyle Y}} \right)}^{{b_i} - 1}}}}
\end{equation}

\noindent and then inserted into equation~(\ref{eq:CDEP}):

\begin{equation}
\label{eq:alphaCal2}
\sum\limits_{i = 1}^n {{\alpha_i}{u^{{e_i}{b_i}}}{{\left( {\frac{{{p_i}}}{Y}} \right)}^{{b_i}}}}
= {\alpha_1}{u^{{e_1}{b_1}}}\frac{{{b_1}}}{{{s_1}}}{\left( {\frac{{{p_1}}}{Y}} \right)^{{b_1}}}
\left[ {\sum\limits_{i = 1}^n {\frac{{{s_i}}}{{{b_i}}}} } \right] \equiv 1
\end{equation}

\noindent The final expression in equation~(\ref{eq:alphaCal2}) can be used to solve for
$\alpha_1$ since the formula must equal~1 by definition:

\begin{equation}
{\alpha_1} = {u^{ - {e_1}{b_1}}}\frac{{{s_1}}}{{{b_1}}}
{\left( {\frac{Y}{{{p_1}}}} \right)^{{b_1}}}
{\left[ {\sum\limits_{i = 1}^n {\frac{{{s_i}}}{{{b_i}}}} } \right]^{ - 1}}
\end{equation}

\noindent Substituting back into equation~(\ref{eq:alphaCal2}) we get:

\begin{equation}
{\alpha_i} = \frac{{{x_i}}}{{{b_i}}}{u^{ - {e_i}{b_i}}}
{\left( {\frac{Y}{{{p_i}}}} \right)^{{b_i} - 1}}{\left[ {\sum\limits_{j = 1}^n
{\frac{{{s_j}}}{{{b_j}}}} } \right]^{ - 1}}
\end{equation}

\noindent The final calibration expression is then the following:

\begin{equation}
\label{eq:alphaCal}
{\alpha_i} = \frac{{{s_i}}}{{{b_i}}}{\left( {\frac{Y}{{{p_i}}}} \right)^{{b_i}}}
\frac{{{u^{ - {e_i}{b_i}}}}}{{\sum\limits_{j = 1}^n {\frac{{{s_j}}}{{{b_j}}}} }}
\end{equation}

Utility is undefined in the base data and it is easiest to simply set it to~1.

In conclusion, for calibration we need the budget shares, initial prices, total expenditure,
income elasticities and the own-price uncompensated elasticities. From this, we can derive
base year consumption volumes, the Allen partial substitution elasticities through
equation~(\ref{eq:APEOWN}), $\sigma$ (and therefore $b$) through equation~(\ref{eq:SIGMACAL})
and the inversion of the $A$-matrix, $e$ through equation~(\ref{eq:ECAL}) and inversion of the
$B$-matrix, and finally $\alpha$ through equation~(\ref{eq:alphaCal}).

It is possible that the initial shares and elasticities lead to inconsistent calibrated values
for the $b$ or $e$ parameters. One solution, modified from \cite{Hertel1997}, is to implement some
sort of maximum entropy method---explicitly imposing the constraints on the parameters. Step~1
is to calibrate the $b$-parameters using the following minimization problem:

\begin{displaymath}
\min L = \sum\limits_i {{s_i}{{({\varepsilon_{ii}} - \varepsilon_{ii}^0)}^2}}
\end{displaymath}

\noindent subject to

\begin{displaymath}
{\varepsilon_{ii}} = (1 - {b_i})\;({s_i} - 1) - {s_i}\left[ {{b_i} + {\eta_i}
- \sum\limits_j {{s_j}{b_j}} } \right]
\end{displaymath}

\begin{displaymath}
0 < {b_i} < 1
\end{displaymath}

The loss function is a weighted some of square errors where $\varepsilon^0$ represents the
initial or target own-price elasticity and $\varepsilon$ will be the estimated elasticity with
the constraints holding. The first constraint is a transformation of equation~(\ref{eq:EPS0})
where the income elasticity is substituted into the definition of the own-price elasticity
(swapping out for the yet unknown $e$-coefficients). One critical issue is to ascertain what
income elasticities to use in the formula above. One could use the target income elasticities,
or an initial transformation of the target elasticities such as described above.

The next step calibrates the $e$-parameters with some target income elasticities as given as well
as the now calibrated $b$-parameters. The minimization problem is formulated as the following:

\begin{displaymath}
\min L = \sum\limits_i {{s_i}{{({\eta_i} - \eta_i^0)}^2}}
\end{displaymath}

\noindent subject to

\begin{displaymath}
{\eta_i} = \frac{1}{{\sum\limits_k {{s_k}} {e_k}}}\left[ {{e_i}{b_i}
- \sum\limits_k {{s_k}} {e_k}{b_k}} \right] - ({b_i} - 1) + \sum\limits_k {{b_k}{s_k}}
\end{displaymath}

\begin{displaymath}
\sum\limits_i {{s_i}{\eta_i}}  \equiv 1
\end{displaymath}

\begin{displaymath}
\left( {{\eta_i} - 1} \right)\;\left( {\eta_i^0 - 1} \right) > 0
\end{displaymath}

The final constraint insures that the estimated income elasticities preserve their relationship
relative to~1, i.e. target elasticities lower than~1 remain lower than~1 in the
estimation procedure.

\subsection{CDE in first differences}

It is useful to decompose changes in demand using a linearized version of the
demand function, and that which is used in the standard GEMPACK version of the CDE function.
The CDE implicit utility function can be used to derive a relation between changes in
income, utility and prices (all in per capita terms). The first step in the
differentiation of the utility function, equation~(\ref{eq:CDEV}), leads to the
following expression:

\[\begin{array}{*{20}{l}}
0& = &{\sum\limits_i {\displaystyle {\alpha_i}{e_i}{b_i}{u^{{e_i}{b_i} - 1}}{{\left( {\frac{{{p_i}}}{Y}} \right)}^{{b_i}}}du} }\\
{}& - &{\sum\limits_i {\displaystyle{\alpha_i}{b_i}{u^{{e_i}{b_i}}}{{\left( {\frac{p_i}{Y}} \right)}^{{b_i} - 1}}\frac{{{p_i}}}{{{Y^2}}}dY} }\\
{}& + &{\sum\limits_i {\displaystyle{\alpha_i}{b_i}{u^{{e_i}{b_i}}}{{\left( {\frac{p_i}{Y}} \right)}^{{b_i} - 1}}\frac{1}{Y}d{p_i}} }
\end{array}\]

\noindent This can be simplified by inserting the expression for the demand equation, equation~(\ref{eq:CDEP}),
and replacing demand with the budget shares ($s_i$):

\[0 = \frac{{du}}{u}\sum\limits_i {{e_i}{s_i}}  - \frac{{dY}}{Y}\sum\limits_i {{s_i}}  + \sum\limits_i {{s_i}\frac{{d{p_i}}}{{{p_i}}}}
\]

\noindent And the final expression can be written as:

\begin{equation}
\label{eq:YPCT}
\overset{.}{Y}=\sum_i{e_is_i}\overset{.}{u}+\sum_i{s_i\overset{.}{p_i}}
\end{equation}

\noindent where the dotted variables represent the percent change (and noting that the sum of the budget shares is equal to 1).

The differentiation of the demand function, equation~(\ref{eq:CDEP}) is somewhat more tedious. The first step
leads to the following expression:

\[\begin{array}{*{20}{l}}
{d{x_i}}& = &\displaystyle {{\alpha _i}{b_i}{e_i}{b_i}{u^{{e_i}{b_i} - 1}}{{\left( {\frac{{{p_i}}}{Y}} \right)}^{{b_i} - 1}}\frac{{du}}{D}}\\
{}& + &\displaystyle {{\alpha _i}{b_i}{u^{{e_i}{b_i}}}\left( {{b_i} - 1} \right){{\left( {\frac{{{p_i}}}{Y}} \right)}^{{b_i} - 2}}\frac{1}{Y}\frac{{d{p_i}}}{D}}\\
{}& - &\displaystyle {{\alpha _i}{b_i}{u^{{e_i}{b_i}}}\left( {{b_i} - 1} \right){{\left( {\frac{{{p_i}}}{Y}} \right)}^{{b_i} - 2}}\frac{{{p_i}}}{{{Y^2}}}\frac{{dY}}{D}}\\
{}& - &\displaystyle {{\alpha _i}{b_i}{u^{{e_i}{b_i}}}{{\left( {\frac{{{p_i}}}{Y}} \right)}^{{b_i} - 1}}{D^{ - 2}}\sum\limits_j {{\alpha _j}{b_j}{e_j}{b_j}{u^{{e_j}{b_j} - 1}}{{\left( {\frac{{{p_j}}}{Y}} \right)}^{{b_j}}}du} }\\
{}& - &\displaystyle {{\alpha _i}{b_i}{u^{{e_i}{b_i}}}{{\left( {\frac{{{p_i}}}{Y}} \right)}^{{b_i} - 1}}{D^{ - 2}}\sum\limits_j {{\alpha _j}{b_j}{b_j}{u^{{e_j}{b_j}}}{{\left( {\frac{{{p_j}}}{Y}} \right)}^{{b_j} - 1}}\frac{1}{Y}d{p_j}} }\\
{}& + &\displaystyle {{\alpha _i}{b_i}{u^{{e_i}{b_i}}}{{\left( {\frac{{{p_i}}}{Y}} \right)}^{{b_i} - 1}}{D^{ - 2}}\sum\limits_j {{\alpha _j}{b_j}{b_j}{u^{{e_j}{b_j}}}{{\left( {\frac{{{p_j}}}{Y}} \right)}^{{b_j} - 1}}\frac{{{p_j}}}{{{Y^2}}}dY} }
\end{array}\]

\noindent where $D$ is the denominator in the demand equation. This can be simplified to the following expression in terms of the percent changes:

\[\begin{array}{*{20}{l}}
\overset{.}{x_i}& = &\displaystyle
e_i b_i \overset{.}{u} +
\left( b_i - 1 \right) \overset{.}{p_i} -
\left( b_i - 1 \right) \overset{.}{Y} \\
{}& - &\displaystyle
\sum\limits_j {e_j b_j s_j} \overset{.}{u}  -
\sum\limits_j {b_j s_j \overset{.}{p_j}} +
\sum\limits_j {b_j s_j} \overset{.}{Y}
\end{array}\]

\noindent Re-grouping terms, the expression becomes:

\[\begin{array}{*{20}{l}}
\overset{.}{x_i}& = &(b_i - 1)\overset{.}{p_i} - \sum\limits_j {b_j s_j \overset{.}{p_j}} \\
{}& + & \overset{.}{u}\left[ e_i b_i - \sum\limits_j {e_j b_j s_j } \right]\\
{}& + & \overset{.}{Y}\left[ \sum\limits_j {b_j s_j}  - \left( b_i - 1 \right) \right]
\end{array}\]

\noindent The percent change in $u$ can be replaced with the expression above, equation~(\ref{eq:YPCT}),
to yield the following after re-arrangement:

\[\begin{array}{*{20}{l}}
\overset{.}{x_i} & = & \displaystyle
\left(b_i - 1\right)\overset{.}{p_i} -
\sum\limits_j {b_j s_j \overset{.}{p_j}}  - \frac{1}{\sum\limits_k {e_k s_k}}
\sum\limits_j {s_j \overset{.}{p_j} \left[ e_i b_i - \sum\limits_k {{e_k}{b_k}{s_k}} \right]} \\
{}& + & \displaystyle
\overset{.}{Y}\left[ \sum\limits_k {b_k s_k}  - \left( b_i - 1 \right)
+ \frac{1}{\sum\limits_k {e_k s_k }} \left( e_i b_i - \sum\limits_k {e_k b_k s_k}  \right) \right]
\end{array}\]

\noindent The final formula inserts the formulas for the income and price elasticities from above to
simplify further to the following expression:

\begin{equation}
\label{eq:XPCT}
\overset{.}{x_i} = \displaystyle
\sum_j{\varepsilon_{ij}\overset{.}{p_j}} + \eta_i \overset{.}{Y}
\end{equation}
