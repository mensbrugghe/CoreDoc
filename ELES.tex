% The ELES description

Many models assume separability in household decision making between saving and current consumption.
Lluch and Howe\footnote{See \cite{Lluch1973} and \cite{Howe1975}.} introduced a relatively
straightforward extension of the LES consumer demand function to include the saving decision
simultaneously with the allocation of income on goods and services, this has become known as the
extended linear expenditure system or the ELES. The ELES is based on consumers maximizing their
intertemporal utility between a bundle of current consumption and an expected future consumption
bundle represented in the form of savings.

\subsection{Basic formulation}

The utility function of the ELES has the following form:

\begin{equation}
\label{eq:ELESU}
u = \prod\limits_i {\left( x_i - \gamma _i \right)^{\mu_i}}
   \left( \frac {S} {P^s} \right)^{\mu_s}
\end{equation}

\noindent with

\begin{equation}
\label{eq:ELESCONSTRAINT}
\sum\limits_i {\mu _i}  + \mu _s = 1
\end{equation}

\noindent where $u$ is utility, $x$ is the vector of consumption goods, $S$ is household saving
(in value), $P^s$ is the price of saving, and $\mu$ and $\gamma$ are ELES parameters.

The consumer solves the following problem:

\[
\max \prod\limits_i {\left( x_i - \gamma _i \right)^{\mu_i}}
   \left( \frac {S} {P^s} \right)^{\mu_s}
\]

\noindent subject to

\[\sum\limits_{i = 1}^n {{p_i}{x_i}}  + S = Y\]

\noindent where $p$ is the vector of consumer prices, and $Y$ is disposable income.
The demand functions are:

\begin{equation}
\label{eq:ELESDEMAND}
x_i = \gamma_i + \frac {\mu_i} {p_i}
   \left( Y - \sum\limits_{j = 1}^n {p_j  \gamma_j} \right)
\end{equation}

\begin{equation}
\label{eq:ELESSAV}
S = \mu_s \left( Y - \sum\limits_{j = 1}^n {p_j \gamma_j } \right) =
   Y - \sum\limits_{j = 1}^n {p_j  x_j}
\end{equation}

The term in parentheses is sometimes called supernumerary income, i.e. it is the income that
remains after subtracting total expenditures on the so-called subsistence (or floor) expenditures
as represented by the $\gamma$  parameter. The parameter $\mu$ then represents the marginal budget
share out of supernumerary income.

\subsection{ELES elasticities}

From the demand equation we can derive the income and price elasticities:

\begin{equation}
\label{eq:ELESINCELAS}
\eta_i = \frac {\mu_i {Y}} {p_i x_i} = \frac {\mu _i} {s_i}   \quad
   \eta _s = \frac {\mu_s {Y}} {S} = \frac{ \mu_s} {s}
\end{equation}

\begin{equation}
\label{eq:ELESOWNPELAS}
\varepsilon_i = \frac {\gamma_i \left( 1 - \mu _i \right)} {x_i} - 1   \quad
   \varepsilon_s =  - 1
\end{equation}

\begin{equation}
\label{eq:ELESXPELAS}
\varepsilon_{ij} =  - \frac {\mu_i p_j \gamma_j } {p_i x_i} =
   - \frac{\mu_i p_j \gamma_j } {s_i Y}   \quad
   \varepsilon_{sj} =  - \frac{\mu_s p_j \gamma_j } {sY} =
      - \frac{p_j \gamma_j } {Y^*}
\end{equation}

\noindent where $s$ is the average propensity to save. Note that the matrix of elasticities can be
collapsed to a single formula using the Kronecker factor:

\begin{equation}
\label{eq:ELESPELAS}
\varepsilon_{ij} = -\frac{\mu_i p_j \gamma_j}{p_i x_i} -
      \delta_{ij} \frac{p_i x_i - p_i \gamma_i}{p_i x_i}  =
   - \frac{\mu_i} {s_i Y} \left[ \delta_{ij} Y^* + p_j \gamma _j \right] =
   - \eta_i \left[ \delta_{ij} \frac{Y^*}{Y} + \frac{p_j \gamma _j}{Y} \right]
\end{equation}

\noindent The last expression shows that there is clear linkage between the income and price
elasticities. At the limit, when income is much larger than supernumerary income, the two
are virtually identical in levels (with opposite signs).

\subsection{Welfare}

With the addition of saving, the indirect utility function is given by:

\begin{equation}
\label{eq:ELESINDU}
v(p,Y) = \prod\limits_i { \left( \frac {\mu_i} {p_i} {Y^*} \right)^{\mu _i}}
   \left( \frac {\mu_s} {P^s} {Y^*} \right)^{\mu _s}
\end{equation}

\noindent or

\begin{equation}
\label{eq:ELESINDU2}
v(p,Y) = \frac{Y^*}{P}
\end{equation}

\noindent where

\[
P = \prod\limits_i {\left( \frac{p_i}{\mu_i} \right)^{\mu _i} }
   \left( \frac{P^s}{\mu _s} \right)^{\mu _s}
\]

The expenditure function is derived by minimizing the cost of achieving a given level of
utility, $u$. It is set-up as:

\[\min \sum\limits_{i = 1}^n {{p_i}{x_i}}  + S\]

\noindent subject to
\[
\prod\limits_i {\left( x_i - \gamma _i \right)^{\mu _i}}
   \left( \frac {S} {P^s} \right)^{\mu _s} = u
\]

\noindent The final expression for the expenditure function is:

\begin{equation}
\label{eq:ELESEXP}
   E\left( p,u \right) = \sum\limits_{i = 1}^n {{p_i}{\gamma _i}}  + uP
\end{equation}

\noindent where $P$, the aggregate price index (including the price of savings) is defined
as above.

\subsection{Calibration}

Calibration of the ELES uses the budget share information from the base SAM, including the
household saving share. Typically, calibration uses income elasticities for all of the $n$
commodities represented in the demand system and uses equation~(\ref{eq:ELESINCELAS}) to derive
the marginal budget shares, $\mu_i$. This procedure leads to a residual income elasticity, which
in this case is the income elasticity of saving. The derived savings income elasticity may be
implausible, in which case adjustments need to be made to individual income elasticities for the
goods, or adjustments can be made on the group of goods, assuming some target for the savings
income elasticity.

The first step is therefore to calculate the marginal budget shares using the average budget
shares and the initial income elasticity estimates.

\[
   \mu _i = \frac {\eta_i p_i x_i} {Y} = {\eta _i}{s_i}
\]

The savings marginal budget share is derived from the consistency requirement that the marginal
budget shares sum to 1:

\[
   \mu_s = 1 - \sum\limits_{i = 1}^n {{\mu _i}}
\]

Assuming this procedure leads to a plausible estimate for the savings income elasticity, the next
step is to calibrate the subsistence minima, $\gamma$. This can be done by seeing that the demand
equations, (\ref{eq:ELESDEMAND}), are linear in the $\gamma$ parameters. Note that in the case of
the ELES the system of equation are of full rank because the $\mu_i$ parameters do not sum to 1
(over the $n$ commodities)---they only sum to 1 including the marginal saving share.\footnote{Note
that the calibration and the setup of the ELES assume explicitly that the minimal expenditure on
savings is zero.}
This may lead to calibration problems if the propensity to save is 0, which may be the case in
some SAMs with poor households. The linear system can be written as:

\[
   C = I\gamma  + MY - M \Pi \gamma
\]

\noindent where $I$ is an $n \times n$ identity matrix, $M$ is a diagonal matrix with
$\mu_i / P_i$ on the diagonal, and $\Pi$ is a matrix, where each row is identical, each row being
the transpose of the price vector. The above system of linear equations can be solved via matrix
inversion for the parameter $\gamma$:

\[
\gamma = A^{-1} C^*
\]

\noindent where

\[A = I - M\ \Pi \]
\[{C^*} = C - MY\]

The matrices $A$ and $C^*$ are defined by:

\[
   A = \left[ {{a_{ij}}} \right] = \left[ \delta_{ij} -  \mu_i\frac{p_j}{p_i}\right] =
   \left\{ {\begin{array}{*{20}{l}}
   {1 - {\mu _i}} & {{\rm{if}}} & {i = j}\\
   { - {\mu _i} \frac{\displaystyle p_j} {\displaystyle p_i}} & {{\rm{if}}} & {i \ne j}
\end{array}} \right.
\]

\[{C^*} = \left[ {{c_i}} \right] = {x_i} - \frac{{{\mu _i}Y}}{{{p_i}}}\]

\noindent The $A$ and $C^*$ matrices are simplified if the price vector is initialized at 1:

\[A = \left[ {{a_{ij}}} \right] = \left\{ {\begin{array}{*{20}{l}}
{1 - {\mu _i}}&{{\rm{if}}}&{i = j}\\
{ - {\mu _i}}&{{\rm{if}}}&{i \ne j}
\end{array}} \right.\]

\[{C^*} = \left[ {{c_i}} \right] = {x_i} - {\mu _i}Y\]

In GAMS one could invert the system of equations embodied in equation~(\ref{eq:ELESDEMAND}) directly
by solving for the endogenous $\gamma$ while holding all of the other variables and parameters
fixed.
