% The AIDADS description

Many commonly used utility functions typically exhibit poor Engel behavior---particularly
in a dynamic framework. The CDE utility function, popularized by the GTAP model
(see \cite{Hertel1997}), has relatively constant income elasticities. The LES utility
function has even worse behavior, as in the absence of any shifts in the
underlying parameters, the LES converges relatively quickly to a Cobb-Douglas utility
function as rapidly rising consumption tends to dominate the floor consumption parameters,
even when adjusting the latter to take into account population growth.
Rimmer and Powell (see \cite{RimmerPowell1992A}, \cite{RimmerPowell1992B} and \cite{RimmerPowell1996})
examine an extension to the standard LES demand system that in effect allows the marginal
propensity to consumer parameter to be driven by changes in utility. Their utility function
has been called An Implicitly Direct Additive Demand System, or AIDADS. The LES function
is a special case of the AIDADS system where the marginal propensity variable is constant.
This extension allows for more complex demand behavior, as well as providing better validation for
observed changes in consumption patterns.\footnote{AIDADS has also been explored in the context
of the GTAP model, see for example \cite{Yuetal2003}.}

\subsection{Basic formulation}

AIDADS starts with the implicitly additive utility function given by:

\begin{equation}
\label{eq:AIDADSPRIMAL}
\sum\limits_i {{U_i}\left( {{x_i},u} \right) \equiv 1}
\end{equation}

\noindent Assume the following functional form for the utility function:

\begin{equation}
\label{eq:AIDADSU}
{U_i} = {\mu _i}\ln \left( {\frac{{{x_i} - {\gamma _i}}}{{A{e^u}}}} \right)
\end{equation}

\noindent where

\begin{equation}
\label{eq:AIDADSMPC}
\mu_i = \frac{{{\alpha_i} + {\beta_i}G\left( u \right)}}{{1 + G\left( u \right)}}
\end{equation}

\noindent with the restrictions

\[\sum\limits_i {{\alpha _i}}  = \sum\limits_i {{\beta _i}}  = 1\]
\[0 \le {\alpha _i} \le 1\]
\[0 \le {\beta _i} \le 1\]
\[{\gamma _i} < {x_i}\]

Cost minimization implies the following:
\[\min \sum\limits_i {{p_i}{x_i}} \]

\noindent subject to

\begin{equation}
\label{eq:AIDADSPRIMAL2}
\sum\limits_i {{\mu _i}\ln \left( {\frac{{{x_i} - {\gamma _i}}}{{A{e^u}}}} \right) \equiv 1}
 \end{equation}

The first order conditions lead to:

\begin{equation}
\label{eq:AIDADSFOC1}
\lambda \frac{{\partial {U_i}}}{{\partial {x_i}}} = {p_i} =
   \lambda \frac{{{\mu _i}}}{{{x_i} - {\gamma _i}}} \Rightarrow \lambda {\mu _i} =
   {p_i}{x_i} - {p_i}{\gamma _i}
\end{equation}

\noindent Taking the sum over $i$ and using the fact that the $\mu_i$ sum to unity implies:

\begin{equation}
\label{eq:AIDADSFOC2}
\lambda  = \sum\limits_i {{p_i}{x_i}}  - \sum\limits_i {{p_i}{\gamma _i}}
   = Y - \sum\limits_i {{p_i}{\gamma _i}}  = {Y^*}
\end{equation}

\noindent where $Y$ is aggregate expenditure, and $Y^*$, sometimes referred to as supernumerary
income, is residual expenditure after subtracting total expenditure on the so-called
subsistence minima, $\gamma$.

Re-inserting equation~(\ref{eq:AIDADSFOC2}) into~(\ref{eq:AIDADSFOC1}) yields the consumer demand
equations:

\begin{equation}
\label{eq:AIDADSDEM}
{x_i} = {\gamma _i} + \frac{{{\mu _i}}}{{{p_i}}}{Y^*} =
   {\gamma _i} + \frac{{{\mu _i}}}{{{p_i}}}\left[ {Y - \sum\limits_j {{p_j}{\gamma _j}} } \right]
\end{equation}

Equation~(\ref{eq:AIDADSDEM}) is virtually identical to the LES demand equation.
Similar to the linear expenditure system (LES), demand is the sum of two components---a
subsistence minimum, $\gamma$, and a share, $\mu$, of supernumerary income. Unlike the LES,
the share parameter, $\mu$, is not constant, but depends on the level of utility itself.
AIDADS collapses to the LES if each $\alpha$ parameter is equal to the corresponding $\beta$
parameter, with the ensuing function of utility, $G(u)$, dropping from
equation~(\ref{eq:AIDADSMPC}).

\subsection{Elasticities}

This section develops the main expressions for the income and price elasticities.
These formulas will be needed to calibrate the initial parameters of the AIDADS function.

\subsubsection{Income elasticities}

To derive further properties of AIDADS requires specifying a functional form for $G(u)$.
\cite{RimmerPowell1996} propose the following:


\begin{equation}
\label{eq:AIDADSG}
G(u) = {e^u}
\end{equation}

The first step is to calculate the marginal budget share, $\rho$ , defined as:

\[{\rho _i} = {p_i}\frac{{\partial {x_i}}}{{\partial Y}}\]

The following expression can be derived from equation~(\ref{eq:AIDADSDEM}):

\[\frac{{\partial {x_i}}}{{\partial Y}} =
   \frac{{{Y^*}}}{{{p_i}}}\frac{{\partial {\mu _i}}}{{\partial Y}} +
   \frac{{{\mu _i}}}{{{p_i}}}\frac{{\partial {Y^*}}}{{\partial Y}} =
   \frac{{{Y^*}}}{{{p_i}}}\frac{{\partial {\mu _i}}}{{\partial u}}\frac{{\partial u}}{{\partial Y}}
   + \frac{{{\mu _i}}}{{{p_i}}}
\]

\noindent Thus:

\begin{equation}
\label{eq:AIDADSRHO}
\rho_i = \mu_i + {Y^*}\frac{{\partial {\mu _i}}}{{\partial u}}\frac{{\partial u}}{{\partial Y}}
\end{equation}

Expression~(\ref{eq:AIDADSRHO}) can be expanded in two steps---first evaluating the partial
derivative of the share variable, $\mu$, with respect to utility, and then the more difficult
calculation of the partial derivative of $u$ with respect to income. The marginal share formula is:

\[{\mu _i} = \frac{{{\alpha _i} + {\beta _i}{e^u}}}{{1 + {e^u}}}\]

\noindent Its derivative is:

\begin{equation}
\label{eq:AIDADSDMU}
\frac{{\partial {\mu _i}}}{{\partial u}} = \frac{{\left( {1 + {e^u}} \right)
   \left( {{\beta _i}{e^u}} \right) - \left( {{\alpha _i} + {\beta _i}{e^u}} \right)
   {e^u}}}{{{{(1 + {e^u})}^2}}} = \frac{{{e^u}\left( {{\beta _i} - {\alpha _i}} \right)}}
{{{{(1 + {e^u})}^2}}}
\end{equation}

Utility and income are combined in implicit form and thus we will invoke the implicit function
theorem to calculate the partial derivative of $u$ with respect to $Y$. First, insert
equation~(\ref{eq:AIDADSDEM}) into equation~(\ref{eq:AIDADSPRIMAL2}):

\[\sum\limits_i {{\mu _i}\ln \left( {\frac{{{x_i} - {\gamma _i}}}{{A{e^u}}}} \right) =
   \sum\limits_i {{\mu _i}\ln \left( {\frac{{{\mu _i}{Y^*}}}{{A{e^u}{p_i}}}} \right) = } 1}
\]

\noindent Expanding the latter expression yields:

\begin{equation}
\label{eq:AIDADSIF1}
f\left( {u,Y} \right) = \sum\limits_i {{\mu _i}\ln \left( {\frac{{{\mu _i}}}{{{p_i}}}} \right)}
   + \ln \left( {{Y^*}} \right) - \ln \left( A \right) - u = 1
\end{equation}

\noindent which provides the implicit relation between $Y$ and $u$. The implicit function theorem
states the following:

\begin{equation}
\label{eq:AIDADSIF2}
\frac{{\partial u}}{{\partial Y}} =  - \frac{{\partial f}}{{\partial Y}}
   {\left[ {\frac{{\partial f}}{{\partial u}}} \right]^{ - 1}}
\end{equation}

\noindent The partial derivative of $f$ with respect to $Y$ is simply:

\begin{equation}
\label{eq:AIDADSIF3}
\frac{{\partial f}}{{\partial Y}} = \frac{1}{{{Y^*}}}
\end{equation}

\noindent The partial derivative of $f$ with respect to $u$ is:

\begin{equation}
\label{eq:AIDADSIF4}
\begin{array}{*{20}{l}}
{\frac{\displaystyle{\partial f}}{\displaystyle{\partial u}}}& = &{ - 1 +
   \sum\limits_i {\displaystyle \left[ {\frac{{\partial {\mu _i}}}{{\partial u}}
   \ln \left( {\frac{\displaystyle{{\mu _i}}}{\displaystyle{{p_i}}}} \right) +
   {\mu _i}\frac{{{p_i}}}{{{\mu _i}}}{p_i}\frac{{\partial {\mu _i}}}{{\partial u}}} \right]} }\\
{}& = &{ - 1 + \frac{\displaystyle{{e^u}}}{\displaystyle{{{(1 + {e^u})}^2}}}\sum\limits_i
   {\left[ {\left( {\ln \left( {\frac{\displaystyle{{\mu _i}}}{\displaystyle{{p_i}}}} \right) + 1}
   \right)\left( {{\beta _i} - {\alpha _i}} \right)} \right]} }\\
{}& = &{\frac{\displaystyle{{e^u}}}{\displaystyle{{{(1 + {e^u})}^2}}}
   \left[ {\sum\limits_i {\displaystyle\left( {{\beta _i} - {\alpha _i}} \right)
   \ln \left( {{x_i} - {\gamma _i}} \right) - \frac{{{{(1 + {e^u})}^2}}}{{{e^u}}}} } \right]}\\
{}& = &{\frac{\displaystyle{\displaystyle{e^u}}}
   {\displaystyle{{{(1 + {e^u})}^2}}}{\Omega ^{ - 1}}}
\end{array}
\end{equation}

\noindent where

\begin{equation}
\label{eq:AIDADSOMEGA}
\Omega  = {\left[ {\sum\limits_i {\left( {{\beta _i} - {\alpha _i}} \right)
   \ln \left( {{x_i} - {\gamma _i}} \right) -
   \frac{{{{\left( {1 + {e^u}} \right)}^2}}}{{{e^u}}}} } \right]^{\; - 1}}
\end{equation}

\noindent The second line uses equation~(\ref{eq:AIDADSDMU}).
In the third line, equation~(\ref{eq:AIDADSDEM})
substitutes for the expression in the logarithm, and the adding up constraint allows for the
deletion of non-indexed variables. Substituting equations~(\ref{eq:AIDADSIF3})
and~(\ref{eq:AIDADSIF4}) into equation~(\ref{eq:AIDADSIF2}) yields:

\begin{equation}
\label{eq:AIDADSIF5}
\frac{{\partial u}}{{\partial Y}} =
- \frac{\Omega }{{{Y^*}}}\frac{{{{\left( {1 + {e^u}} \right)}^2}}}{{{e^u}}}
\end{equation}

\noindent Substituting equations~(\ref{eq:AIDADSDMU}) and~(\ref{eq:AIDADSIF5}) into
equation~(\ref{eq:AIDADSRHO}) yields the following expression for $\rho$:

\[
{\rho _i} = {\mu _i} - \left( {{\beta _i} - {\alpha _i}} \right)\;\Omega
\]

\noindent The income elasticities are derived from the following expression:

\[
{\eta _i} = \frac{{\partial {x_i}}}{{\partial Y}}\frac{Y}{{{x_i}}} =
   \frac{{\partial {x_i}}}{{\partial Y}}\frac{Y}{{{x_i}}}
   \frac{{{p_i}}}{{{p_i}}} = \frac{{{\rho _i}}}{{{s_i}}}
\]

\noindent where $s_i$ is the average budget share:

\[
{s_i} = \frac{{{p_i}{x_i}}}{Y} = \frac{{{p_i}{\gamma _i}}}{Y} + {\mu _i}\frac{{{Y^*}}}{Y}
   = {\mu _i} + \left( {\frac{{{p_i}{\gamma _i}
   - {\mu _i}\sum\limits_j {{p_j}{\gamma _j}} }}{Y}} \right)
\]

\noindent Thus the income elasticity, $\eta$, is equal to the ratio of the marginal budget
share, $\rho$, and the average budget share, $s$. Finally, equation~(\ref{eq:AIDADSETA})
describes one formulation of the income elasticity:

\begin{equation}
\label{eq:AIDADSETA}
\eta _i =\frac{\mu_i - \left( \beta_i - \alpha_i \right) \Omega} {s_i}
\end{equation}

\subsubsection{Price elasticity}

The matrix of substitution elasticities is identical to the expression for the LES and has the form:

\begin{equation}
\label{eq:AIDADSAPE}
\sigma_{ij} = \left[ {{\mu_j} - {\delta_{ij}}} \right]
   \frac{{{\mu_i}{Y^*}}}{{{s_i}{s_j}Y}}
\end{equation}

\noindent where $\delta$ is the Kronecker delta:

\[
{\delta _{ij}} = \left\{ {\begin{array}{*{20}{c}}
1&{i = j}\\
0&{i \ne j}
\end{array}} \right.
\]

\noindent It is clear that the matrix is symmetric.
The matrix of substitution elasticities is also equal to:

\[{\sigma _{ij}} = \left[ {{\mu _j} - {\delta _{ij}}} \right]
   \frac{{{\mu _i}{Y^*}}}{{{s_i}{s_j}Y}} = \frac{{\left( {{x_i} - {\gamma _i}} \right)}}
   {{{x_i}}}\frac{{\left( {{x_j} - {\gamma _j}} \right)}}{{{x_j}}}\frac{Y}{{{Y^*}}} -
   \frac{{{\delta _{ij}}}}{{{s_j}}}\frac{{\left( {{x_i} - {\gamma _i}} \right)}}{{{x_i}}}
\]

The compensated demand elasticities derive from the following:

\begin{equation}
\label{eq:AIDADSCDE}
{\xi _{ij}} = {s_j}{\sigma _{ij}} = \left[ {{\mu _j} - {\delta _{ij}}} \right]
   \frac{{{\mu _i}{Y^*}}}{{{s_i}Y}}
\end{equation}

Finally, the matrix of uncompensated demand elasticities is given by:

\begin{equation}
\label{eq:AIDADSUED1}
{\varepsilon _{ij}} = {\xi _{ij}} - {s_j}{\eta _i} =
   \left[ {{\mu _j} - {\delta _{ij}}} \right]\;\frac{{{\mu _i}{Y^*}}}{{{s_i}Y}} - {s_j}{\eta _i}
\end{equation}

\noindent The uncompensated demand elasticities can also be written as:

\begin{equation}
\label{eq:AIDADSUED2}
{\varepsilon _{ij}} =  - \frac{{{\mu _i}}}{{{s_i}Y}}\left[ {{p_j}{\gamma _j} +
   {\delta _{ij}}{Y^*}} \right] +
   \frac{{{s_j}}}{{{s_i}}}\left( {{\beta _i} - {\alpha _i}} \right)\Omega
\end{equation}

The first term on the right-hand side is always negative. The second term differs from the LES
expression for the uncompensated demand elasticities.\footnote{Recall that for the LES, the
$\alpha$ and $\beta$ terms are equal and thus the second term drops.}  We can see from
expression~(\ref{eq:AIDADSUED2}) that the AIDADS specification allows for both gross
complementarity and substitution. As well, it allows for luxury goods, i.e. positive own-price
demand elasticities should the second term be positive and greater than the first term.

\subsection{Implementation}

Implementation of AIDADS is somewhat more complicated than the LES since the marginal propensity to
 consume out of supernumerary income is endogenous, and utility is defined implicitly.
The following four equations are needed for model implementation:

\begin{equation}
\label{eq:AIDADSIMP1}
{Y^*} = Y - \sum\limits_i {{p_i}{\gamma _i}}
\end{equation}

\begin{equation}
\label{eq:AIDADSIMP2}
{x_i} = {\gamma _i} + \frac{{{\mu _i}}}{{{p_i}}}{Y^*}
\end{equation}

\begin{equation}
\label{eq:AIDADSIMP3}
{\mu _i} = \frac{{{\alpha _i} + {\beta _i}{e^u}}}{{1 + {e^u}}}
\end{equation}

\begin{equation}
\label{eq:AIDADSIMP4}
u = \sum\limits_i {{\mu _i}\ln \left( {{x_i} - {\gamma _i}} \right)}  - 1 - \ln \left( A \right)
\end{equation}

Equations~(\ref{eq:AIDADSIMP1}) and~(\ref{eq:AIDADSIMP2}) are identical to their LES and ELES
counterparts.\footnote{Though the definition of $Y$ includes
savings in the case of the ELES.}
Equation~(\ref{eq:AIDADSIMP3}) determines the level of the marginal propensity to
consume out of supernumerary income, $\mu$, which is a constant in the case of the LES (ELES).
It requires however the calculation of the utility level, $u$, which is defined in
equation~(\ref{eq:AIDADSIMP4}).

\subsection{Calibration}

[To be updated] Calibration requires more information than the LES. Where the LES has $2n$
parameters to calibrate (subject to consistency constraints), AIDADS has $3n$ parameters
(less the consistency requirements)---$\alpha$, $\beta$ and $\gamma$. The calibration system
includes equations~(\ref{eq:AIDADSIMP1}) through~(\ref{eq:AIDADSIMP4}) which have $2+2n$ endogenous
variables ($Y^*$, $\gamma$, $\mu$, and $A$). There are no equations for calibrating the $\alpha$
and $\beta$ parameters. If we have knowledge of the income elasticities, we can add the following
equations:

\begin{equation}
\label{eq:AIDADSIMP5}
\Psi  = \frac{1}{\Omega } = \left[ {\sum\limits_i {\left( {{\beta _i} - {\alpha _i}} \right)
   \ln \left( {{x_i} - {\gamma _i}} \right) -
   \frac{{{{\left( {1 + {e^u}} \right)}^2}}}{{{e^u}}}} } \right]
\end{equation}

\begin{equation}
\label{eq:AIDADSIMP6}
{\eta _i} = \frac{{{\rho _i}}}{{{s_i}}} = \frac{{{\mu _i} - \left( {{\beta _i}
- {\alpha _i}} \right) \Omega }}{{{s_i}}} = \frac{{{\mu _i}}}{{{s_i}}} -
   \frac{{\left( {{\beta _i} - {\alpha _i}} \right)}}{{{s_i}\Psi }}
\end{equation}

There are an additional $1+n$ equations, solving for $\Psi$ and $\alpha$. There is need for an
additional $n$ equations. Assuming we have knowledge of at least $n$ price elasticities, for example
the own-price elasticities, we can add the following equation:

\begin{equation}
\label{eq:AIDADSIMP7}
{\varepsilon _{ii}} =  - \frac{{{\mu _i}}}{{{s_i}Y}}\left[ {{p_i}{\gamma _i} + {Y^*}} \right] +
   \left( {{\beta _i} - {\alpha _i}} \right)\Omega
\end{equation}

The $\alpha$ and $\beta$ parameters are not independent, the following restrictions must hold:

\begin{equation}
\label{eq:AIDADSIMP8}
\sum\limits_i {{\alpha_i} = 1}
\end{equation}

\begin{equation}
\label{eq:AIDADSIMP9}
\sum\limits_i {{\beta_i} = 1}
\end{equation}

The system is under-determined, there are $5+4n$ equations and $3+4n$ variables. One solution, is
to make the own-price elasticities endogenous. In this case, we are adding $n$ variables, but then
the system is over-determined. We can minimize a loss function with respect to the
price elasticities:

\[L = \sum\limits_i {{{\left( {{\varepsilon _i} - \varepsilon _i^0} \right)}^2}} \]

\noindent where $\varepsilon^0$ represents an initial guess of the own-price elasticities and the
calibration algorithm will calculate the endogenous $\varepsilon$ in order to minimize the loss
function, subject to constraints~(\ref{eq:AIDADSIMP5}) through~(\ref{eq:AIDADSIMP9}) and the model
equations~(\ref{eq:AIDADSIMP1}) through~(\ref{eq:AIDADSIMP4}). The exogenous parameters in the
calibration procedure include $p$, $x$, $s$, $Y$, $\eta$, $\varepsilon^0$ and $u$.
